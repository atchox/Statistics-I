\documentclass[12pt,letterpaper,fleqn]{article}
\usepackage{fullpage}
\usepackage[top=2cm, bottom=4.5cm, left=2.5cm, right=2.5cm]{geometry}
\usepackage{amsmath,amsthm,amsfonts,amssymb,amscd}
\usepackage{lastpage}
\usepackage{enumerate}
\usepackage{fancyhdr}
\usepackage{mathrsfs}
\usepackage{xcolor}
\usepackage{graphicx}
\usepackage{listings}
\usepackage{hyperref}
\usepackage[T1]{fontenc}
\usepackage{textcomp}

\hypersetup{
  colorlinks=true,
  linkcolor=blue,
  linkbordercolor={0 0 1}
}

\definecolor{limitblue}{RGB}{32, 76, 113}
\definecolor{ruddybrown}{rgb}{0.73, 0.4, 0.16}
\colorlet{punct}{red!60!black} 
\definecolor{background}{HTML}{EEEEEE}
\definecolor{delim}{RGB}{20,105,176}
\definecolor{ogreen}{rgb}{0.0, 0.5, 0.0}
\colorlet{numb}{magenta!60!black}
 
\renewcommand\lstlistingname{Code}
\renewcommand\lstlistlistingname{Codes}
\def\lstlistingautorefname{Alg.}

\lstdefinestyle{Python}{
  language        = Python,
  frame           = lines, 
  basicstyle      = \footnotesize,
  keywordstyle    = \color{blue},
  stringstyle     = \color{green},
  commentstyle    = \color{red}\ttfamily
}
\lstdefinestyle{R}{
  language        = R,
  frame           = lines,
  captionpos      = b,
  abovecaptionskip= 10pt, 
  emphstyle       = \textbf,
  framextopmargin = 4pt,
  framexbottommargin = 4pt,
  basicstyle      = \ttfamily\footnotesize,
  keywordstyle    = \color{limitblue},
  stringstyle     = \color{ruddybrown},
  showstringspaces= false,
  commentstyle    = \color{red}\ttfamily,
  tabsize         = 2,
  literate=
    *{0}{{{\color{numb}0}}}{1}
      {1}{{{\color{numb}1}}}{1}
      {2}{{{\color{numb}2}}}{1}
      {3}{{{\color{numb}3}}}{1}
      {4}{{{\color{numb}4}}}{1}
      {5}{{{\color{numb}5}}}{1}
      {6}{{{\color{numb}6}}}{1}
      {7}{{{\color{numb}7}}}{1}
      {8}{{{\color{numb}8}}}{1}
      {9}{{{\color{numb}9}}}{1}
      {:}{{{\color{punct}{:}}}}{1}
      {,}{{{\color{punct}{,}}}}{1}
      {\{}{{{\color{delim}{\{}}}}{1}
      {\}}{{{\color{delim}{\}}}}}{1}
      {[}{{{\color{delim}{[}}}}{1}
      {]}{{{\color{delim}{]}}}}{1}
}

\setlength{\parindent}{0.0in}
\setlength{\parskip}{0.05in}

% Edit these as appropriate
\newcommand\course{Statistics I}
\newcommand\hwnumber{6.1}                  % <-- homework number
\newcommand\NetIDa{Atreya Choudhury}           % <-- NetID of person #1
\newcommand\NetIDb{bmat2005}           % <-- NetID of person #2 (Comment this line out for problem sets)

\pagestyle{fancyplain}
\headheight 35pt
\lhead{\NetIDa}
\lhead{\NetIDa\\\NetIDb}                 % <-- Comment this line out for problem sets (make sure you are person #1)
\chead{\textbf{\Large Assignment \hwnumber}}
\rhead{\course \\ \today}
\lfoot{}
\cfoot{}
\rfoot{\small\thepage}
\headsep 2em

\begin{document}
\begin{flushleft}
  \begin{equation}
    \begin{split}
      &\hspace{1.4em}E[(\hat{\theta} - \theta)^2]\\
      &= E[\hat{\theta}^2] - 2\theta E[\hat{\theta}^2] + \theta ^2\\
      &= (E[\hat{\theta}]^2 - 2\theta E[\hat{\theta}^2] + \theta ^2) + (E[\hat{\theta}^2] - [\hat{\theta}]^2)\\
      &= (E[\hat{\theta}] - \theta)^2 + Variance\\
      &= Bias^2 + Variance\\
    \end{split}
  \end{equation}
  We find the mean squared error of X
  \begin{equation}
    \begin{split}
      &\hspace{1.4em}Bias[X]\\
      &= E[X] - \sum_{i=1}^n p_i\\
      &= E\Biggl[\sum_{i=1}^n X_i\Biggr] - \sum_{i=1}^n p_i\\
      &= \sum_{i=1}^n (E[X_i] - p_i)\\
      &= \sum_{i=1}^n 0\\
      &= 0
    \end{split}
  \end{equation}
  \begin{equation}
    \begin{split}
      &\hspace{1.4em}Variance[X]\\
      &= E[X^2] - E[X]^2\\
      &= E\Biggl[\Biggl(\sum_{i=1}^n X_i\Biggr)^2\Biggr] - E\Biggl[\sum_{i=1}^n X_i\Biggr]^2\\
      &= E\Biggl[\sum_{i=1}^n {X_i}^2 + \sum_{i \neq j} X_i X_j \Biggr] - \Biggl(\sum_{i=1}^n p_i\Biggr)^2\\
      &= E\Biggl[\sum_{i=1}^n {X_i}^2\Biggr] + E\Biggl[\sum_{i \neq j} X_i X_j \Biggr] - \Biggl(\sum_{i=1}^n p_i\Biggr)^2\\
      &= \sum_{i=1}^n p_i + \sum_{i \neq j} E[X_i]E[X_j] - \Biggl(\sum_{i=1}^n p_i\Biggr)^2 \hfill [\because X_i's\text{ are independent}]\\
      &= \sum_{i=1}^n p_i + \sum_{i \neq j} p_i p_j - \Biggl(\sum_{i=1}^n p_i\Biggr)^2\\
      &= \sum_{i=1}^n p_i + \sum_{i \neq j} p_i p_j - \sum_{i=1}^n p_i^2 - \sum_{i \neq j} p_i p_j\\
      &= \sum_{i=1}^n p_i - \sum_{i=1}^n p_i^2\\
      &= n \overline{p} - \sum_{i=1}^n p_i^2
    \end{split}
  \end{equation}
  We find the mean squared error of Y
  \begin{equation}
    \begin{split}
      &\hspace{1.4em}Bias[Y]\\
      &= E[Y] - \sum_{i=1}^n p_i\\
      &= n\overline{p} - n\overline{p}\\
      &= 0
    \end{split}
  \end{equation}
  \begin{equation}
    \begin{split}
      &\hspace{1.4em}Variance[Y]\\
      &= E[Y^2] - E[Y]^2\\
      &= n\overline{p} - n\overline{p}^2
    \end{split}
  \end{equation}
  \begin{equation}
    \begin{split}
      &\hspace{1.4em}MSE[X]\\
      &= n \overline{p} - \sum_{i=1}^n p_i^2\\
      &\leq n \overline{p} - n \biggl(\sum_{i=1}^n p_i\biggr)^2 \hfill[\text{Cauchy-Schwarz Inequality}]\\
      &\leq n \overline{p} - n \overline{p}^2\\
      &\leq n \overline{p} - n \overline{p}^2\\
      &\leq MSE[Y]
    \end{split}
  \end{equation}
  $\therefore$ X is a better estimator than Y.
\end{flushleft}
\end{document}