\documentclass[12pt,letterpaper]{article}
\usepackage{fullpage}
\usepackage[top=2cm, bottom=4.5cm, left=2.5cm, right=2.5cm]{geometry}
\usepackage{amsmath,amsthm,amsfonts,amssymb,amscd}
\usepackage{lastpage}
\usepackage{enumerate}
\usepackage{fancyhdr}
\usepackage{mathrsfs}
\usepackage{xcolor}
\usepackage{graphicx}
\usepackage{listings}
\usepackage{hyperref}
\usepackage[T1]{fontenc}
\usepackage{textcomp}

\hypersetup{
  colorlinks=true,
  linkcolor=blue,
  linkbordercolor={0 0 1}
}

\definecolor{limitblue}{RGB}{32, 76, 113}
\definecolor{ruddybrown}{rgb}{0.73, 0.4, 0.16}
\colorlet{punct}{red!60!black} 
\definecolor{background}{HTML}{EEEEEE}
\definecolor{delim}{RGB}{20,105,176}
\definecolor{ogreen}{rgb}{0.0, 0.5, 0.0}
\colorlet{numb}{magenta!60!black}
 
\renewcommand\lstlistingname{Code}
\renewcommand\lstlistlistingname{Codes}
\def\lstlistingautorefname{Alg.}

\lstdefinestyle{Python}{
  language        = Python,
  frame           = lines, 
  basicstyle      = \footnotesize,
  keywordstyle    = \color{blue},
  stringstyle     = \color{green},
  commentstyle    = \color{red}\ttfamily
}
\lstdefinestyle{R}{
  language        = R,
  frame           = lines,
  captionpos      = b,
  abovecaptionskip= 10pt, 
  emphstyle       = \textbf,
  framextopmargin = 4pt,
  framexbottommargin = 4pt,
  basicstyle      = \ttfamily\footnotesize,
  keywordstyle    = \color{limitblue},
  stringstyle     = \color{ruddybrown},
  showstringspaces= false,
  commentstyle    = \color{red}\ttfamily,
  tabsize         = 2,
  literate=
    *{0}{{{\color{numb}0}}}{1}
      {1}{{{\color{numb}1}}}{1}
      {2}{{{\color{numb}2}}}{1}
      {3}{{{\color{numb}3}}}{1}
      {4}{{{\color{numb}4}}}{1}
      {5}{{{\color{numb}5}}}{1}
      {6}{{{\color{numb}6}}}{1}
      {7}{{{\color{numb}7}}}{1}
      {8}{{{\color{numb}8}}}{1}
      {9}{{{\color{numb}9}}}{1}
      {:}{{{\color{punct}{:}}}}{1}
      {,}{{{\color{punct}{,}}}}{1}
      {\{}{{{\color{delim}{\{}}}}{1}
      {\}}{{{\color{delim}{\}}}}}{1}
      {[}{{{\color{delim}{[}}}}{1}
      {]}{{{\color{delim}{]}}}}{1}
}

\setlength{\parindent}{0.0in}
\setlength{\parskip}{0.05in}

% Edit these as appropriate
\newcommand\course{Statistics I}
\newcommand\hwnumber{9.1}                  % <-- homework number
\newcommand\NetIDa{Atreya Choudhury}           % <-- NetID of person #1
\newcommand\NetIDb{bmat2005}           % <-- NetID of person #2 (Comment this line out for problem sets)

\pagestyle{fancyplain}
\headheight 35pt
\lhead{\NetIDa}
\lhead{\NetIDa\\\NetIDb}                 % <-- Comment this line out for problem sets (make sure you are person #1)
\chead{\textbf{\Large Assignment \hwnumber}}
\rhead{\course \\ \today}
\lfoot{}
\cfoot{}
\rfoot{\small\thepage}
\headsep 2em

\begin{document}
\textit{An automobile parts store was interested in comparing the mean life length of three
brands of automobile brake shoes. The following data represents the life length,
measured in 1,000’s of miles, of random samples of six sets of brake shoes of each
brand:}
\begin{center}
  \begin{tabular}{ |c|c|c| }
    \hline
    Brakes1 & Brakes2 & Brakes3\\
    \hline
    43 & 51 & 34\\
    44 & 65 & 45\\
    41 & 67 & 37\\
    47 & 58 & 48\\
    54 & 57 & 38\\
    \hline
  \end{tabular}
\end{center}

\begin{enumerate}[a.] \setlength{\itemsep}{30pt}
  \item State the null and alternative hypotheses to test whether there is a significant
  difference in mean life length among the three brands of brake shoes. Let $\alpha = 0.05$.
  
  \textbf{Solution:}

  $H_0$ : $\mu_0 = \mu_1 = \mu_2$\\
  $H_a$ : At least 2 $\mu_i$'s are different
  \item Compute the ANOVA table.

  \textbf{Solution:}

  \begin{table}[h]
    \centering
    \begin{tabular}{ c|ccccc }
      & Df & Sum Sq & Mean Sq & F value & Pr(>F)\\
      \hline
      brakes & 2 & 1203.4 & 601.7 & 17.42 & 0.000123\\
      Residuals & 15 & 518.2 & 34.5 & & \\
    \end{tabular}
    \caption{ANOVA Table}
  \end{table}
  \item Perform the test using the p-value approach.

  \textbf{Solution:}

  The p-value is equal to 0.000123 which is less than $\alpha = 0.05$.
  Hence, we reject the null hypothesis.
  \item What is the conclusion for the three brands of brakes?

  \textbf{Solution:}
  
  We have a statistical difference in the mean life lengths among the three brakes.
\end{enumerate}
\newpage
\lstinputlisting[language=R, style=R, title=R Code]{code.R}
\end{document}