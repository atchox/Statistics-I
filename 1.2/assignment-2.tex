\documentclass[12pt,letterpaper]{article}
\usepackage{fullpage}
\usepackage[top=2cm, bottom=4.5cm, left=2.5cm, right=2.5cm]{geometry}
\usepackage{amsmath,amsthm,amsfonts,amssymb,amscd}
\usepackage{lastpage}
\usepackage{enumerate}
\usepackage{fancyhdr}
\usepackage{mathrsfs}
\usepackage{xcolor}
\usepackage{graphicx}
\usepackage{listings}
\usepackage{hyperref}

\hypersetup{%
  colorlinks=true,
  linkcolor=blue,
  linkbordercolor={0 0 1}
}
 
\renewcommand\lstlistingname{Algorithm}
\renewcommand\lstlistlistingname{Algorithms}
\def\lstlistingautorefname{Alg.}

\lstdefinestyle{Python}{
    language        = Python,
    frame           = lines, 
    basicstyle      = \footnotesize,
    keywordstyle    = \color{blue},
    stringstyle     = \color{green},
    commentstyle    = \color{red}\ttfamily
}

\setlength{\parindent}{0.0in}
\setlength{\parskip}{0.05in}

% Edit these as appropriate
\newcommand\course{Statistics I}
\newcommand\hwnumber{1.2}                  % <-- homework number
\newcommand\NetIDa{Atreya Choudhury}           % <-- NetID of person #1
\newcommand\NetIDb{bmat2005}           % <-- NetID of person #2 (Comment this line out for problem sets)

\pagestyle{fancyplain}
\headheight 35pt
\lhead{\NetIDa}
\lhead{\NetIDa\\\NetIDb}                 % <-- Comment this line out for problem sets (make sure you are person #1)
\chead{\textbf{\Large Assignment \hwnumber}}
\rhead{\course \\ \today}
\lfoot{}
\cfoot{}
\rfoot{\small\thepage}
\headsep 2em

\begin{document}
\textbf{Question:}
\textit{In each of the following cases, comment if there is selection, response or nonresponse bias. Give reasons. How would you avoid these biases in each case?}

\begin{enumerate}[1.] \setlength{\itemsep}{30pt}
  \item \textit{The quantity of interest is average time spent on the internet for persons with internet access. The question is posted in surveymonkey and volunteers respond.}

  \textbf{\textit{Answer: }}
  
  This study is subject to non response bias.

  We can assume that volunteers on \textit{surveymonkey} are people who spend quite some time on the internet.
  Daily users usually do not go around filling surveys.
  So, we miss out on casual users who like to, say, google the occasional phrase or scroll through their social media feed.

  The survey should be sneaked into popular sites which see a lot of daily traffic.
  A \textbf{small} survey on Instagram or FaceBook or, say, on top of a Google search results page would be more effective.
  Filling surveys can be incentivised by offering vouchers or lottery tickets.
  \item \textit{For collecting data on how supportive the population is on gay marriages, a journalist surveys participants in a pride parade.}
  
  \textbf{\textit{Answer: }}
  
  This study is subject to selection bias.

  Most people participating in the pride parade are supportive of gay marriages.
  People who are not supportive will not attend the parade.
  The journalist will miss out on the survey of people who are not supportive of gay marriages.

  The journalist should use anonymous submissions choosing people at random from a \textbf{general} population.
  It is essential that the submissions should be anonymous or else that could be cause for response bias.
  
  \item \textit{To measure the prevalence of cheating in college, students of a college are asked if they ever cheated in any exam.}
  
  \textbf{\textit{Answer: }}

  This study is subject to response bias.

  It is a sensitive question and people who have actually cheated in exams will not say that they have cheated.
  The people who have not cheated will definitely not say that they have.
  This would cause the data to be extremely skewed.

  The process should be made as anonymous as possible, like say anonymous submissions into a secure box.

\end{enumerate}
\end{document}