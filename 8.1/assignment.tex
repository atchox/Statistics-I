\documentclass[12pt,letterpaper]{article}
\usepackage{fullpage}
\usepackage[top=2cm, bottom=4.5cm, left=2.5cm, right=2.5cm]{geometry}
\usepackage{amsmath,amsthm,amsfonts,amssymb,amscd}
\usepackage{lastpage}
\usepackage{enumerate}
\usepackage{fancyhdr}
\usepackage{mathrsfs}
\usepackage{xcolor}
\usepackage{graphicx}
\usepackage{listings}
\usepackage{hyperref}
\usepackage[T1]{fontenc}
\usepackage{textcomp}

\hypersetup{
  colorlinks=true,
  linkcolor=blue,
  linkbordercolor={0 0 1}
}

\definecolor{limitblue}{RGB}{32, 76, 113}
\definecolor{ruddybrown}{rgb}{0.73, 0.4, 0.16}
\colorlet{punct}{red!60!black} 
\definecolor{background}{HTML}{EEEEEE}
\definecolor{delim}{RGB}{20,105,176}
\definecolor{ogreen}{rgb}{0.0, 0.5, 0.0}
\colorlet{numb}{magenta!60!black}
 
\renewcommand\lstlistingname{Code}
\renewcommand\lstlistlistingname{Codes}
\def\lstlistingautorefname{Alg.}

\lstdefinestyle{Python}{
  language        = Python,
  frame           = lines, 
  basicstyle      = \footnotesize,
  keywordstyle    = \color{blue},
  stringstyle     = \color{green},
  commentstyle    = \color{red}\ttfamily
}
\lstdefinestyle{R}{
  language        = R,
  frame           = lines,
  captionpos      = b,
  abovecaptionskip= 10pt, 
  emphstyle       = \textbf,
  framextopmargin = 4pt,
  framexbottommargin = 4pt,
  basicstyle      = \ttfamily\footnotesize,
  keywordstyle    = \color{limitblue},
  stringstyle     = \color{ruddybrown},
  showstringspaces= false,
  commentstyle    = \color{red}\ttfamily,
  tabsize         = 2,
  literate=
    *{0}{{{\color{numb}0}}}{1}
      {1}{{{\color{numb}1}}}{1}
      {2}{{{\color{numb}2}}}{1}
      {3}{{{\color{numb}3}}}{1}
      {4}{{{\color{numb}4}}}{1}
      {5}{{{\color{numb}5}}}{1}
      {6}{{{\color{numb}6}}}{1}
      {7}{{{\color{numb}7}}}{1}
      {8}{{{\color{numb}8}}}{1}
      {9}{{{\color{numb}9}}}{1}
      {:}{{{\color{punct}{:}}}}{1}
      {,}{{{\color{punct}{,}}}}{1}
      {\{}{{{\color{delim}{\{}}}}{1}
      {\}}{{{\color{delim}{\}}}}}{1}
      {[}{{{\color{delim}{[}}}}{1}
      {]}{{{\color{delim}{]}}}}{1}
}

\setlength{\parindent}{0.0in}
\setlength{\parskip}{0.05in}

% Edit these as appropriate
\newcommand\course{Statistics I}
\newcommand\hwnumber{8.1}                  % <-- homework number
\newcommand\NetIDa{Atreya Choudhury}           % <-- NetID of person #1
\newcommand\NetIDb{bmat2005}           % <-- NetID of person #2 (Comment this line out for problem sets)

\pagestyle{fancyplain}
\headheight 35pt
\lhead{\NetIDa}
\lhead{\NetIDa\\\NetIDb}                 % <-- Comment this line out for problem sets (make sure you are person #1)
\chead{\textbf{\Large Assignment \hwnumber}}
\rhead{\course \\ \today}
\lfoot{}
\cfoot{}
\rfoot{\small\thepage}
\headsep 2em

\begin{document}
\textit{A certain company of battery claims that the average life of a certain type of their battery is 75 weeks. The average life of each of 9 randomly selected batteries is listed below. Assume the battery life distribution is normal. It is of interest to know if the sample data suggest the average life is smaller than 75 weeks.}
$$74.5, 75.0, 72.3, 76.0, 75.2, 75.1, 75.3, 74.9, 74.8$$

\begin{enumerate}[a.] \setlength{\itemsep}{30pt}
  \item State the appropriate null and alternative hypotheses.
  
  \textbf{Solution:}
  The null hypothesis is that the average life of the company's battery is 75 years and the alternative hypothesis is that the average life of the battery is less than 75 years.
  
  $H_0$ : $\mu = 75$\\
  $H_a$ : $\mu < 75$

  \item Compute the test statistic for the hypotheses in part (a).

  \textbf{Solution:}

  The test statistic is given by

  \begin{equation}
    \begin{split}
      t &= \frac{\overline{x} - \mu}{\frac{s}{\sqrt{n}}}\\
      \implies t &\approx -6.207
    \end{split}
  \end{equation}
  \item Compute the approximate p-value associated with the test statistic in (b). Does the sample data support the alternative hypothesis at the 0.05 level?

  \textbf{Solution:}
  The p-value associated with the test statistic is given by

  \begin{equation}
    \begin{split}
      p &= \mathbb{P}(t < -6.207)\\
      \implies p &\approx 0.276
    \end{split}
  \end{equation}

  $\because \alpha = 0.05 < 0.276,$ we do not have enough evidence to reject the null hypothesis.

  The sample data does not support the alternative hypothesis at the 0.05 level.
  \end{enumerate}
\newpage
\lstinputlisting[language=R, style=R, title=R Code]{code.R}
\end{document}